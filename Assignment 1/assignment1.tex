\PassOptionsToPackage{unicode=true}{hyperref} % options for packages loaded elsewhere
\PassOptionsToPackage{hyphens}{url}
%
\documentclass[]{article}
\usepackage{lmodern}
\usepackage{amssymb,amsmath}
\usepackage{ifxetex,ifluatex}
\usepackage{fixltx2e} % provides \textsubscript
\ifnum 0\ifxetex 1\fi\ifluatex 1\fi=0 % if pdftex
  \usepackage[T1]{fontenc}
  \usepackage[utf8]{inputenc}
  \usepackage{textcomp} % provides euro and other symbols
\else % if luatex or xelatex
  \usepackage{unicode-math}
  \defaultfontfeatures{Ligatures=TeX,Scale=MatchLowercase}
\fi
% use upquote if available, for straight quotes in verbatim environments
\IfFileExists{upquote.sty}{\usepackage{upquote}}{}
% use microtype if available
\IfFileExists{microtype.sty}{%
\usepackage[]{microtype}
\UseMicrotypeSet[protrusion]{basicmath} % disable protrusion for tt fonts
}{}
\IfFileExists{parskip.sty}{%
\usepackage{parskip}
}{% else
\setlength{\parindent}{0pt}
\setlength{\parskip}{6pt plus 2pt minus 1pt}
}
\usepackage{hyperref}
\hypersetup{
            pdftitle={Assignmment 1},
            pdfauthor={Daniel Alonso},
            pdfborder={0 0 0},
            breaklinks=true}
\urlstyle{same}  % don't use monospace font for urls
\usepackage[margin=1in]{geometry}
\usepackage{color}
\usepackage{fancyvrb}
\newcommand{\VerbBar}{|}
\newcommand{\VERB}{\Verb[commandchars=\\\{\}]}
\DefineVerbatimEnvironment{Highlighting}{Verbatim}{commandchars=\\\{\}}
% Add ',fontsize=\small' for more characters per line
\usepackage{framed}
\definecolor{shadecolor}{RGB}{248,248,248}
\newenvironment{Shaded}{\begin{snugshade}}{\end{snugshade}}
\newcommand{\AlertTok}[1]{\textcolor[rgb]{0.94,0.16,0.16}{#1}}
\newcommand{\AnnotationTok}[1]{\textcolor[rgb]{0.56,0.35,0.01}{\textbf{\textit{#1}}}}
\newcommand{\AttributeTok}[1]{\textcolor[rgb]{0.77,0.63,0.00}{#1}}
\newcommand{\BaseNTok}[1]{\textcolor[rgb]{0.00,0.00,0.81}{#1}}
\newcommand{\BuiltInTok}[1]{#1}
\newcommand{\CharTok}[1]{\textcolor[rgb]{0.31,0.60,0.02}{#1}}
\newcommand{\CommentTok}[1]{\textcolor[rgb]{0.56,0.35,0.01}{\textit{#1}}}
\newcommand{\CommentVarTok}[1]{\textcolor[rgb]{0.56,0.35,0.01}{\textbf{\textit{#1}}}}
\newcommand{\ConstantTok}[1]{\textcolor[rgb]{0.00,0.00,0.00}{#1}}
\newcommand{\ControlFlowTok}[1]{\textcolor[rgb]{0.13,0.29,0.53}{\textbf{#1}}}
\newcommand{\DataTypeTok}[1]{\textcolor[rgb]{0.13,0.29,0.53}{#1}}
\newcommand{\DecValTok}[1]{\textcolor[rgb]{0.00,0.00,0.81}{#1}}
\newcommand{\DocumentationTok}[1]{\textcolor[rgb]{0.56,0.35,0.01}{\textbf{\textit{#1}}}}
\newcommand{\ErrorTok}[1]{\textcolor[rgb]{0.64,0.00,0.00}{\textbf{#1}}}
\newcommand{\ExtensionTok}[1]{#1}
\newcommand{\FloatTok}[1]{\textcolor[rgb]{0.00,0.00,0.81}{#1}}
\newcommand{\FunctionTok}[1]{\textcolor[rgb]{0.00,0.00,0.00}{#1}}
\newcommand{\ImportTok}[1]{#1}
\newcommand{\InformationTok}[1]{\textcolor[rgb]{0.56,0.35,0.01}{\textbf{\textit{#1}}}}
\newcommand{\KeywordTok}[1]{\textcolor[rgb]{0.13,0.29,0.53}{\textbf{#1}}}
\newcommand{\NormalTok}[1]{#1}
\newcommand{\OperatorTok}[1]{\textcolor[rgb]{0.81,0.36,0.00}{\textbf{#1}}}
\newcommand{\OtherTok}[1]{\textcolor[rgb]{0.56,0.35,0.01}{#1}}
\newcommand{\PreprocessorTok}[1]{\textcolor[rgb]{0.56,0.35,0.01}{\textit{#1}}}
\newcommand{\RegionMarkerTok}[1]{#1}
\newcommand{\SpecialCharTok}[1]{\textcolor[rgb]{0.00,0.00,0.00}{#1}}
\newcommand{\SpecialStringTok}[1]{\textcolor[rgb]{0.31,0.60,0.02}{#1}}
\newcommand{\StringTok}[1]{\textcolor[rgb]{0.31,0.60,0.02}{#1}}
\newcommand{\VariableTok}[1]{\textcolor[rgb]{0.00,0.00,0.00}{#1}}
\newcommand{\VerbatimStringTok}[1]{\textcolor[rgb]{0.31,0.60,0.02}{#1}}
\newcommand{\WarningTok}[1]{\textcolor[rgb]{0.56,0.35,0.01}{\textbf{\textit{#1}}}}
\usepackage{graphicx,grffile}
\makeatletter
\def\maxwidth{\ifdim\Gin@nat@width>\linewidth\linewidth\else\Gin@nat@width\fi}
\def\maxheight{\ifdim\Gin@nat@height>\textheight\textheight\else\Gin@nat@height\fi}
\makeatother
% Scale images if necessary, so that they will not overflow the page
% margins by default, and it is still possible to overwrite the defaults
% using explicit options in \includegraphics[width, height, ...]{}
\setkeys{Gin}{width=\maxwidth,height=\maxheight,keepaspectratio}
\setlength{\emergencystretch}{3em}  % prevent overfull lines
\providecommand{\tightlist}{%
  \setlength{\itemsep}{0pt}\setlength{\parskip}{0pt}}
\setcounter{secnumdepth}{0}
% Redefines (sub)paragraphs to behave more like sections
\ifx\paragraph\undefined\else
\let\oldparagraph\paragraph
\renewcommand{\paragraph}[1]{\oldparagraph{#1}\mbox{}}
\fi
\ifx\subparagraph\undefined\else
\let\oldsubparagraph\subparagraph
\renewcommand{\subparagraph}[1]{\oldsubparagraph{#1}\mbox{}}
\fi

% set default figure placement to htbp
\makeatletter
\def\fps@figure{htbp}
\makeatother


\title{Assignmment 1}
\author{Daniel Alonso}
\date{November 20th, 2020}

\begin{document}
\maketitle

Importing libraries

\begin{Shaded}
\begin{Highlighting}[]
\KeywordTok{library}\NormalTok{(Rcpp)}
\end{Highlighting}
\end{Shaded}

\hypertarget{teacher-example-this-code-is-not-mine}{%
\subsection{Teacher example (this code is NOT
mine)}\label{teacher-example-this-code-is-not-mine}}

\begin{Shaded}
\begin{Highlighting}[]
\NormalTok{my_knn_R =}\StringTok{ }\ControlFlowTok{function}\NormalTok{(X, X0, y)\{}
  \CommentTok{# X data matrix with input attributes}
  \CommentTok{# y response variable values of instances in X  }
  \CommentTok{# X0 vector of input attributes for prediction}
  
\NormalTok{  nrows =}\StringTok{ }\KeywordTok{nrow}\NormalTok{(X)}
\NormalTok{  ncols =}\StringTok{ }\KeywordTok{ncol}\NormalTok{(X)}
  
  \CommentTok{# One of the instances is going to be the closest one:}
  \CommentTok{#   closest_distance: it is the distance , min_output}
\NormalTok{  closest_distance =}\StringTok{ }\DecValTok{99999999}
\NormalTok{  closest_output =}\StringTok{ }\DecValTok{-1}
\NormalTok{  closest_neighbor =}\StringTok{ }\DecValTok{-1}
  
  \ControlFlowTok{for}\NormalTok{(i }\ControlFlowTok{in} \DecValTok{1}\OperatorTok{:}\NormalTok{nrows)\{}
    
\NormalTok{    distance =}\StringTok{ }\DecValTok{0}
    \ControlFlowTok{for}\NormalTok{(j }\ControlFlowTok{in} \DecValTok{1}\OperatorTok{:}\NormalTok{ncols)\{}
\NormalTok{      difference =}\StringTok{ }\NormalTok{X[i,j]}\OperatorTok{-}\NormalTok{X0[j]}
\NormalTok{      distance =}\StringTok{ }\NormalTok{distance }\OperatorTok{+}\StringTok{ }\NormalTok{difference }\OperatorTok{*}\StringTok{ }\NormalTok{difference}
\NormalTok{    \}}
    
\NormalTok{    distance =}\StringTok{ }\KeywordTok{sqrt}\NormalTok{(distance)}
    
    \ControlFlowTok{if}\NormalTok{(distance }\OperatorTok{<}\StringTok{ }\NormalTok{closest_distance)\{}
\NormalTok{      closest_distance =}\StringTok{ }\NormalTok{distance}
\NormalTok{      closest_output =}\StringTok{ }\NormalTok{y[i]}
\NormalTok{      closest_neighbor =}\StringTok{ }\NormalTok{i}
\NormalTok{    \}}
\NormalTok{  \}}
\NormalTok{  closest_output}
\NormalTok{\}  }
\end{Highlighting}
\end{Shaded}

\hypertarget{testing-teacher-example-this-code-is-not-mine}{%
\subsection{Testing teacher example (This code is NOT
mine)}\label{testing-teacher-example-this-code-is-not-mine}}

\begin{Shaded}
\begin{Highlighting}[]
\CommentTok{# X contains the inputs as a matrix of real numbers}
\KeywordTok{data}\NormalTok{(}\StringTok{"iris"}\NormalTok{)}
\CommentTok{# X contains the input attributes (excluding the class)}
\NormalTok{X <-}\StringTok{ }\NormalTok{iris[,}\OperatorTok{-}\DecValTok{5}\NormalTok{]}
\CommentTok{# y contains the response variable (named medv, a numeric value)}
\NormalTok{y <-}\StringTok{ }\NormalTok{iris[,}\DecValTok{5}\NormalTok{]}

\CommentTok{# From dataframe to matrix}
\NormalTok{X <-}\StringTok{ }\KeywordTok{as.matrix}\NormalTok{(X)}
\CommentTok{# From factor to integer}
\NormalTok{y <-}\StringTok{ }\KeywordTok{as.integer}\NormalTok{(y)}

\CommentTok{# This is the point we want to predict}

\NormalTok{X0 <-}\StringTok{ }\KeywordTok{c}\NormalTok{(}\FloatTok{5.80}\NormalTok{, }\FloatTok{3.00}\NormalTok{, }\FloatTok{4.35}\NormalTok{, }\FloatTok{1.30}\NormalTok{)}

\CommentTok{# Using my_knn and FNN:knn to predict point X0}
\CommentTok{# Using the same number of neighbors, it should be similar (k=1)}
\KeywordTok{my_knn_R}\NormalTok{(X, X0, y)}
\KeywordTok{library}\NormalTok{(FNN)}
\NormalTok{FNN}\OperatorTok{::}\KeywordTok{knn}\NormalTok{(X, }\KeywordTok{matrix}\NormalTok{(X0, }\DataTypeTok{nrow =} \DecValTok{1}\NormalTok{), y, }\DataTypeTok{k=}\DecValTok{1}\NormalTok{)}
\end{Highlighting}
\end{Shaded}

\end{document}
